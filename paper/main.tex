\documentclass{article}
\usepackage[utf8]{inputenc}
\usepackage{authblk}
\usepackage{graphicx}
\usepackage{subfig}
\usepackage{multirow}
\usepackage{hyperref}
\usepackage{lineno}

\title{Freqgen: A tool for generating DNA sequences with specified amino acid, codon, and $k$-mer frequencies}
\author[1,2]{Benjamin D. Lee}
\author[2]{Paul G. Gamble}
\affil[1]{School of Engineering and Applied Sciences, Harvard University, Cambridge, MA}
\affil[2]{Lab41, In-Q-Tel, Menlo Park, CA}
\date{September 2018}

\usepackage[sorting=none]{biblatex}
\addbibresource{references.bib}

\begin{document}
\maketitle


\begin{abstract}
\end{abstract}

\section{Introduction}
A variety of forces affect the distribution of the frequencies of nucleotides in DNA. For example, GC-content, the ratio of the amount of Gs and Cs to As and Ts, is greater in regions with more recombination \cite{Spencer2006}. Generalizing beyond individual nucleotides, the distribution of $k$-mers, subsequences of length $k$, is also neither random nor uniform. These biases have been capitalized upon in genetic engineering \cite{AlSaif2012}, vaccine production \cite{Tulloch2014}, sequence assembly \cite{Compeau2011}, and metagenomics \cite{Perry2010}. 

To further study and take advantage of these biases, the ability to generate sequences \textit{in silico} with given $k$-mer frequencies is vital. Prior work \cite{Liu2016} has shown the possibility of generating sequences with given amino acid usage and GC-content, albeit not with a specified amino acid sequence. Similarly, other software packages \cite{Gaspar2012, Guimaraes2014} have been developed for synthetic gene design, allowing amino acid sequence specification, but they have not focused specifically on matching $k$-mer usage frequencies. Here we introduce Freqgen, a software tool to generate sequences that simultaneously meet amino acid, codon, and $k$-mer usage targets. We call this approach to sequence generation $k$-mer harmonization.

There are two distinct purposes of this tool. The first is in synthetic gene design, in which $k$-mers have a significant role on protein expression \cite{AlSaif2012}. For example, users may wish to recode sequences to simultaneously match the host organism's codon usage bias while minimizing unfavorable dinucelotide frequency. The second use of Freqgen is the generation of novel sequences that \textit{appear} natural for the purposes of data augmentation or as a null model\cite{Liu2016}. In some domains of machine learning such as deep learning, massive amounts of data are required to train models. However, due to genomes' limited sizes, there may not be enough data on which to learn. Freqgen seeks to solve this problem by allowing for the creation of sequences that share the same $k$-mer frequency distribution, codon usage, and amino acid usage as target sequences.

\section{Methods}
We divide our method into two phases: amino acid sequence generation and nucleotide representation generation. This allows users to either generate amino acid sequences from amino acid frequencies or to bypass this step by specifying an amino acid sequence. Then, we use a genetic algorithm, an approach to solving optimization problems based on Darwinian evolution \cite{Vikhar2016}, to create a DNA representation of the amino acid sequence that matches most closely matches $k$-mer and/or codon usage goals.

\subsection{Amino Acid Sequence Generation}
To generate a novel amino acid sequence, we require a frequency for each amino acid which can either be supplied directly or calculated from a set of reference sequences. Then, an amino acid sequence of the desired length is constructed from amino acids chosen with a probability proportional to its frequency.

To illustrate this method, suppose that the desired length is 12 amino acids and a reference set of proteins is $\{QA, QQ\}$. From this we can see that two amino acids, A and Q, occur with 25\% and 75\% frequency in the reference set, respectively. We then choose from $\{A, Q\}$ with a probability of $\{0.25, 0.75\}$. This process is repeated a total of 12 times. Thus, a potential new amino acid sequence would be AAQAQQQQQAQQ. As expected, A comprised 25\% of the amino acids in the sequence with the other 75\% being Qs. This method preserves the overall distribution of the original distribution while having the advantage of being both fast and scaling linearly with the desired length.

\subsection{$k$-mer Harmonization}
The method by which we use genetic algorithms to create DNA representations of amino acid sequences that have matching $k$-mer usage can be summarized as:

\begin{enumerate}
    \item Given target sequence(s), calculate their $k$-mer and/or codon usage biases.
    \item Create a population of $n$ candidate synonymous sequences at random.
    \item Measure each candidate sequence's fitness with respect to the target frequencies.
    \item Create a new generation of synonymous solutions based on the fittest sequences in the last generation, carrying the best solution forward unchanged to prevent a decline in fitness between generations.
    \item For each individual in the new generation, with probability $m$, randomly "mutate" it while preserving synonymity.
    \item Similarly, with probability $c$, recombine two of the new sequences to produce two new synonymous sequences.
    \item Repeat the process until the fitness of the fittest member of the population has stopped improving for $g$ generations.
\end{enumerate}

where $k$, $n$, $m$, $c$, and $g$ are user-configurable parameters.

\subsubsection{Fitness Calculation}
Freqgen supports two similar metrics of candidate sequence fitness: Euclidean distance and Jensen-Shannon divergence as implemented in \cite{dit}. By default, Freqgen uses Euclidean distance for performance reasons. Jensen-Shannon divergence is provided as an alternative fitness metric due the fact that it is bounded by $[0,1]$ whereas Euclidean distance is unbounded and dependent on $k$, therefore making cross-optimization comparison difficult.

\subsubsection{Mutation and Crossover}
To mutate a candidate solution sequence, we choose a codon as well as a different synonymous codon for it, both with equal probability. We then replace the codon with its synonym and return the mutated synonymous sequence.

For crossover between two parents, we choose a codon index at random. Then, we perform synonymous recombination by swapping the sequences beyond that index. For example, if the two sequences for protein sequence MDT* were ATG-GAT-ACT-TGA and ATG-GAC-ACA-TAA and the zero-based index 1 were chosen, then the two resultant sequences would be ATG-GAT-ACA-TAA and ATG-GAC-ACT-TGA. Note that both of these sequences have the same amino acid sequence.

This approach has a significant advantage over the naive approach of randomly changing one nucleotide in the sequence that a traditional genetic algorithm approach based on real world polymorphisms might use. By replacing a randomly chosen codon with a synonymous one, we can ensure that the mutated individual has the same amino acid sequence as before, eliminating the need to both verify amino acid sequence identity and repeat the mutation function again until a synonymous sequence is generated, as a naive nucleotide swap would require.


\section{Results}

As a demonstration of the use of Freqgen, we recoded the sequence of green fluorescent protein (GFP) from \textit{Aequorea victoria} \cite{Prasher1992} to more closely match the $k$-mer frequency distribution of the highly expressed genes of \textit{Escherichia coli} \cite{Puigbo2007}. For the purposes of visualization, we used $k=2$, although there is no technical limitation as to the value(s) of $k$, given sufficient computing resources. Figure 1 shows the initial, target, and resulting 2-mer frequencies of optimization using Freqgen. Freqgen was able to match all 16 2-mer frequencies with an average error of 3.7\% in 191 generations with population size 100. The optimization took 1 minute and 52 seconds using a 2015 MacBook Pro. 

\section{Discussion}

The discussion will go here.

\section{Conclusion}

Freqgen is a promising tool with applicability wherever $k$-mers are used. It is able to match $k$-mer usage, codon usage, and amino acid usage frequencies via genetic algorithms with both a programmatic API and a command-line interface. Freqgen is freely available from the Python Package Index for the most recent stable release, on GitHub (\href{https://www.github.com/Lab41/freqgen}{github.com/Lab41/freqgen}) for the latest development version, from the archived release on Zenodo, and as additional file 1. The documentation is available at \href{https://freqgen.readthedocs.io}{freqgen.readthedocs.io} for all versions and as additional file 2 for the version at the time of release (0.1).

\printbibliography

\end{document}