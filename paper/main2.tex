\documentclass{bioinfo}
\copyrightyear{2018} \pubyear{2018}

\access{Advance Access Publication Date: Day Month Year}
\appnotes{Applications Note}

\begin{document}
\firstpage{1}

\subtitle{Sequence analysis}

\title[Freqgen]{Freqgen: A tool for generating DNA sequences with specified amino acid, codon, and \textit{k}-mer frequencies}
\author[Lee and Gamble]{Benjamin D. Lee$^{\text{\sfb 1,2,3,}*}$ and Paul G. Gamble\,$^{\text{\sfb 3}}$}
\address{$^{\text{\sf 1}}$School of Engineering and Applied Sciences, Harvard University, Cambridge, MA, 02138, USA, \\
$^{\text{\sf 2}}$Department of Genetics, Harvard Medical School, Boston, MA, 02115, USA, and \\ $^{\text{\sf 3}}$Lab41, In-Q-Tel, Menlo Park, CA, 94025, USA}

\corresp{$^\ast$To whom correspondence should be addressed.}

\history{Received on XXXXX; revised on XXXXX; accepted on XXXXX}

\editor{Associate Editor: XXXXXXX}

\abstract{\textbf{Summary:} Text Text Text Text Text Text Text Text Text Text Text Text Text
Text Text Text Text Text Text Text Text Text Text Text Text Text Text Text Text Text Text Text
Text Text Text Text Text Text Text Text Text Text Text Text Text Text Text Text Text Text Text
Text Text Text Text Text Text
Text Text Text Text Text.\\
\textbf{Availability and Implementation:} Freqgen is written in Python 3 and freely available under the permissive MIT license. The open-source code can be downloaded from \href{github.com/Lab41/freqgen}{github.com/Lab41/freqgen} as well as the Python Package Index. The installation instructions and user guide are available at \href{freqgen.readthedocs.org}{freqgen.readthedocs.org}.\\
\textbf{Contact:} \href{benjamindlee@me.com}{benjamindlee@me.com}\\
\textbf{Supplementary information:} Supplementary data are available at \textit{Bioinformatics}
online.}

\maketitle

\section{Introduction}

A variety of forces affect the distribution of the frequencies of nucleotides in DNA. For example, GC-content, the ratio of the amount of Gs and Cs to As and Ts, is greater in regions with more recombination \citep{spencerHumanPolymorphismRecombination2006}. Generalizing beyond individual nucleotides, the distribution of $k$-mers, subsequences of length $k$, is also neither random nor uniform. These biases have been capitalized upon in genetic engineering \citep{al-saifUUUADinucleotide2012}, vaccine production \citep{tullochRNAVirusAttenuation2014], phylogenetics \citep{deorowiczKmerdbInstantEvolutionary2018}, and metagenomics \citep{perryDistinguishingMicrobialGenome2010}.

To further study and take advantage of these biases, the ability to generate sequences in silico with given $k$-mer frequencies is vital. Prior work \citep{liuNullSeqToolGenerating2016} has shown the possibility of generating sequences with given amino acid usage and GC-content, albeit not with a specified amino acid sequence. Similarly, other software packages \citep{gasparEuGeneMaximizingSynthetic2012, guimaraesDTailorAutomatedAnalysis2014} have been developed for synthetic gene design, allowing amino acid sequence specification, but have not focused specifically on matching $k$-mer usage frequencies for $k>2$. Here we introduce Freqgen, a software tool to generate sequences that simultaneously meet amino acid, codon, and $k$-mer usage targets for arbitrary values of $k$. We call this approach to sequence generation $k$-mer optimization, a generalization of the concept of codon optimization.

There are two distinct uses of this tool. The first is in synthetic gene design, in which $k$-mers have a significant role on protein expression \citep{al-saifUUUADinucleotide2012}. For example, users may wish to recode sequences to simultaneously match the host organism's codon usage bias while minimizing unfavorable dinucelotide frequency. In this case, the specific amino acid sequence is known and passed directly to Freqgen. The second use of Freqgen is the generation of novel coding sequences sharing amino acid, codon, and/or $k$-mer frequencies with that of a target set of sequences. For this use case, the amino acid sequence is not known and is generated by Freqgen by choosing amino acids with the same frequency as they appear in the target set and then performing DNA-level $k$-mer optimization on the generated sequence. This application is useful for the purposes of data augmentation or as a null model \citep{liuNullSeqToolGenerating2016}. In some domains of machine learning such as deep learning, massive amounts of data are required to train models. However, due to genomes' limited sizes, there may not be enough data on which to learn. Freqgen seeks to solve this problem by allowing for the creation of sequences that share the same $k$-mer frequency distribution, codon usage, and amino acid usage as target sequences.

\section{Implementation}

Freqgen accomplishes this optimization by means of a genetic algorithm, which seeks to identify the sequence of codons that have $k$-mer and/or codon frequencies most similar to a set of target frequencies, as defined by minimizing either the Euclidean distance or Jensen-Shannon divergence, as implemented in dit\citep{jamesDitPythonPackage2018}, between the sets. By operating on the codon level as opposed to the nucleotide level, Freqgen is able to ensure that the amino acid sequence is held constant, greatly speeding up the search of the solution space.

\section{Results}


\section{Conclusion}
% \vspace*{-10pt}


\section*{Acknowledgements}

Text Text Text Text Text Text  Text Text.  \citealp{Liu2016} might want to know about  text
text text text
% \vspace*{-12pt}

\section*{Funding}

This work has been supported by the... Text Text  Text Text.
% \vspace*{-12pt}

\bibliographystyle{natbib}
\bibliographystyle{achemnat}
\bibliographystyle{plainnat}
\bibliographystyle{abbrv}
\bibliographystyle{bioinformatics}

\bibliographystyle{plain}

\bibliography{references}

\end{document}
